\documentclass[11pt]{article}

\input{structure.tex}

\hypersetup{
  pdftitle={David Heath - Curriculum vitae},
  pdfauthor={David Heath}
}

\usepackage{bibentry}
\usepackage{natbib}
\usepackage{enumitem}
\usepackage{pdfsync}

\pagenumbering{gobble}

\newcommand{\CC}{C\nolinebreak\hspace{-.05em}\raisebox{.4ex}{\tiny\bf +}\nolinebreak\hspace{-.10em}\raisebox{.4ex}{\tiny\bf +}}
\newcommand{\C}{C}

\nobibliography*


\begin{document}

{\LARGE\bfseries David Heath}
\medskip\medskip

1101 Juniper Street NE\\
Unit 211\\
Atlanta, Georgia, 30309
\medskip\\
Phone: 770-361-6450\\
Email: \href{mailto:daheath@illinois.edu}{daheath@illinois.edu}

\section*{Area of Expertise}

Cryptography; Secure Multiparty Computation


\section*{Employment}

\years{August 2022 -}{\textbf{Assistant Professor}}, University of Illinois Urbana-Champaign
\\

\years{2014-2016}{\textbf{Research Engineer I}, Georgia Tech Research Institute, Atlanta, Georgia
% As part of GTRI's Electronic Systems Laboratory, I worked with
% safety-critical software systems and helped to verify the Future
% Airborne Capability Environment (FACE\texttrademark) Technical
% Standard.



\section*{Earned Degrees}

\years{2016-2022}
\textbf{PhD in Computer Science}, Georgia Institute of Technology, Atlanta, Georgia\\
Advisor: Vladimir Kolesnikov
\medskip\\
\years{2010-2014}
\textbf{BS in Computer Science}\\
\textbf{BS in Mechanical Engineering}, Georgia Institute of Technology, Atlanta, Georgia


\section*{Research Experience}

\years{2018-2022}
\textbf{Graduate Research Assistant}, Georgia Institute of Technology, Atlanta, Georgia\\
Advisor: Vladimir Kolesnikov
\begin{itemize}
    \item I co-discovered ``stacked garbling'', a fundamental improvement to the Garbled Circuit cryptographic primitive.
      Stacked garbling greatly accelerates the secure handling of programs with conditional branching.
      Subsequently, I found similar improvements to other secure computation protocols.
    \item I co-discovered ``one-hot garbling'', a fundamental improvement to the Garbled Circuit cryptographic primitive.
      One-hot garbling greatly accelerates the secure handling of vector operations.
    \item I improved the efficiency of ``garbled RAM'' by multiple orders of magnitude.
      Garbled RAM allows the secure handling of random access arrays.
    \item I developed crypto-technical improvements to interactive Zero Knowledge.
      These improvements culminated in a system that handles proofs expressed
      as off-the-shelf \C\ programs and runs them in the $10$KHz range.
\end{itemize}
\years{2016-2018}
\textbf{Graduate Research Assistant}, Georgia Institute of Technology, Atlanta, Georgia\\
Advisor: William Harris
\begin{itemize}
  \item I co-developed ``Shara'', a new solver for ``Constrained Horn Clause'' (CHC) systems.
    CHC systems can be used to formalize programs; solvers for such systems can prove interesting properties of programs.
  \item I co-developed ``Pequod'', a solver that automatically deduces the equivalence of programs.
\end{itemize}

\section*{Teaching}

\years{2023}\textbf{Spring, CS598DH Secure Computation}
\years{2022}\textbf{Fall, CS598DH Secure Computation}

\section*{Awards, Grants, and Experience on Sponsored Projects}

\years{2022}\textbf{Best Paper Award in IACR Eurocrypt 2022}

\years{2020-2021}\textbf{Institute for Information Security and Privacy Cybersecurity Seed Funding}\\
Principal Investigator: Vladimir Kolesnikov\\
Georgia Tech's Institute for Information Security offers \$50,000 in funding to support promising research in the areas of cybersecurity.
I drafted and collaborated on a submission with my advisor Vladimir Kolesnikov.
The resulting submission``ZK for Anything and Anyone: Practical Zero Knowledge execution of arbitrary C programs'' was funded.
\medskip\\


\years{2019-2020}\textbf{IARPA HECTOR Project}\\
IARPA sponsored HECTOR (Homomorphic Encryption Computing Techniques with
Overhead Reduction), a multi-million dollar research project aimed at improving the usability of secure computation techniques.
As a member of the PANTHEON team, I directly worked on the design and implementation of MPC protocols and language design.
\medskip\\


\years{2016-2020}\textbf{Georgia Tech President's Fellowship}\\
I received a fellowship that Georgia Tech offers to the top 10 percent of Ph.D. applicants.
\medskip\\


\years{2016}\textbf{CS 7001 Research Project Award}\\
Every Computer Science PhD student at Georgia Tech is required to
take CS 7001, an introductory course to academic research. Each
student is required to write and present work featuring research
tasks conducted during the semester. I was presented an award for
best research project as part of the Georgia Tech College of
Computing Annual Awards and Honors ceremony.

\section*{Teaching Experience}

\years{Fall 2019}
\textbf{Guest Lecturer}, Special Topics: Secure Multiparty Computation\\
I gave two one-hour lectures in this graduate level special topics course.
In the first, I presented our new results on Stacked Garbling, both to share interesting new results and to convey a flavor of the research process.
In the second, I presented the EMP Toolkit, a state-of-the-art implementation of many multiparty computation, to share
how it solves problems with real-world \CC\ code.
\medskip\\


\years{Spring 2019}
\textbf{Graduate Teaching Assistant}, Special Topics: Blockchain\\
I generated course materials, held office hours, and graded homeworks and exams for this crosslisted special topics course on blockchain technologies.
This was the first year this course was offered at Georgia Tech, so as a TA I helped develop assignments and exams from scratch.
\medskip\\

\pagebreak
\years{Spring 2018}
\textbf{Graduate Teaching Assistant}, Compilers and Interpreters\\
I generated course materials, held office hours, graded homeworks and exams, and gave one lecture for this undergraduate introduction to compiler technologies.


\section*{Conference Publications}

\begin{description}
  \item \years{2022}
    \bibentry{EC:HeaKolOst22}
  \item
    \bibentry{EC:HHKLOS22}
  \item
    \bibentry{YHKD22}

  \item \years{2021}
    \bibentry{CCS:HK21}
  \item
    \bibentry{AC:HK21}
  \item
    \bibentry{AC:HKP21}
  \item
    \bibentry{EC:HeaKol21}
  \item
    \bibentry{SP:HYDK21}
  \item
    \bibentry{PKC:HeaKolPec21}
  \item
    \bibentry{HKL21}

  \item \years{2020}
    \bibentry{AC:HeaKolPec20}
  \item
    \bibentry{CCS:HeaKol20}
  \item
    \bibentry{C:HeaKol20}
  \item
    \bibentry{EC:HeaKol20}

  \item \years{2019}
    \bibentry{zhou_heath_harris_2019}

  \item \years{2018}
    \bibentry{zhou_heath_harris_2018}
\end{description}

\section*{Ph.D. Dissertation}

\begin{description}
  \item \years{2022}
    \bibentry{PHD:Heath22}
\end{description}

\section*{Unpublished Manuscripts}

\begin{description}
  \item
    \bibentry{visa}
  \item
    \bibentry{symphony}
\end{description}


\section*{Invited Lectures}

\sloppypar
\begin{description}
  \item \years{2022}
    \bibentry{NYCrypto:Heath2022}
  \item
    \bibentry{TPMPC:Heath2022}
  \item
    \bibentry{CharlesRiver:Heath2022}
  \item \years{2021}
    \bibentry{Berkeley:Heath2021}
  \item
    \bibentry{CMU:Heath2021}
  \item
    \bibentry{UMD:Heath2021}
  \item
    \bibentry{Stanford:Heath2021_2}
  \item
    \bibentry{TCC:Heath2021}
  \item
    \bibentry{Stanford:Heath2021}
  \item
    \bibentry{CLS:Heath2021}
  \item \years{2020}
    \bibentry{Stanford:Heath2020}
  \item
    \bibentry{Berkeley:Heath2020}
  \item \years{2019}
    \bibentry{CLS:Heath2019}
\end{description}


\section*{Professional Contributions}

\begin{description}
  \item \years{2023}
    Program Committee Member, Crypto 2023.
  \item 
    Program Committee Member, PKC 2023. 
  \item \years{2022}
    Program Committee Member, ASIACRYPT 2022. 
  \item
    Program Committee Member, CSCML 2022. 
  \item \years{2021}
    Program Committee Member, CCS 2021. 
  \item
    Program Committee Member, CSCML 2021.
  \item \years{2020}
    Program Committee Member, CSCML 2020.
\end{description}


\section*{Open Source Repositories}

\begin{description}
  \item \bibentry{one-hot-repo}
  \item \bibentry{logstack-repo}
  \item \bibentry{proram-repo}
\end{description}



\bibliographystyle{alpha}
\nobibliography{custom,bib/abbrev3,bib/crypto}

\end{document}

